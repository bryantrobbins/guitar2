Model-based software testing excels at the generation of short test cases. Often,
the collective group of test cases (called the test suite) provides a guarantee
of coverage. Consider various existing work by Memon et al., which generates test
cases for event-driven systems such as those driven by Graphical User Interfaces
(GUIs). Their sequence length coverage generation technique experiences an
exponential growth of possible test cases as longer sequence lengths are considered. This
growth often leads to a practical limit of considering only sequences of length 2 or 3.
when generating test suites. Other research suggests that longer test cases, and test
cases involving the interaction of 3 or more variables, find signficantly more
bugs than test cases designed to cover only 1 or 2 input variables in combination.

We suggest that test cases generated from model-based techniques can potentially be
combined in interesting ways while still preserving the desired qualities of the overall
test suite. Significantly reducing the runtime of generated suites by combination would allow
for the consideration of increasingly higher-order test cases.

In support of this goal of combining test cases, in the paper we describe and
evaluate a method for predicting the feasibility of potential test case
combinations. We require our technique
