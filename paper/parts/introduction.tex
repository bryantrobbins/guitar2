Model-based software testing excels at the generation of short test cases. Often,
the collective group of test cases (called the test suite) provides a guarantee
of coverage. Consider various existing work by Memon et al., which generates test
cases for event-driven systems such as those driven by Graphical User Interfaces
(GUIs). Their sequence length coverage generation technique experiences an
exponential growth of possible test cases as longer sequence lengths are considered. This
growth often leads to a practical limit of considering only coverage of sequences 
of length 2 or 3 when generating test suites. Other research suggests that longer
test cases can more efficiently and more effectively find bugs than shorter test
cases.

In this paper, we lay the groundwork for constructing longer test cases from the shorter
test cases typical of model-based testing techniques. We explore whether shorter test cases
can be combined in interesting ways to produce longer test cases while still preserving
desired characteristics. As a first step, we consider the preservation of test case
feasibility. More precisely, we explore whether the execution results of model-based test
cases can be used to predict the feasibility of potential test case combinations. We believe
that this work can later be extended to the prediction of additional test suite properties,
and eventually to generalized reduction techniques for model-based test suites; but
we do not explore those extensions in this work.

