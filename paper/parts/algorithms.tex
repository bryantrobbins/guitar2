In this section, we formalize the algorithms used in our experimental study.
Our high-level goal is to develop an algorithm \textsc{Reduce} which will produce
a reduced test suite from a provided input suite while controlling for two confidence
parameters. We accomplish the reduction through test case combination.

Starting from a copy of the initial test suite $I$, we select and remove a random test case
$i$ from $I$. In the \textsc{FindOne} algorithm, we loop over the possible combinations of 
$i$ with other test cases from $I$, considering the possible ways which $i$ and this additional
test case $t$ could be combined. The enumeration of possible combinations of any two test cases
is given by \textsc{FindPossible}. \textsc{FindOne} returns the first test case $c$ discovered which also
meets both the feasibility probability constraint $\alpha$ and the coverage probability constraint $\beta$.
In the event that we find no such test case $c$, a null value is returned to \textsc{Reduce}, indicating
that the originally considered test case $i$ should be added directly to the reduce test suite $R$ and
no longer considered as a candidate for combination with other test cases remaining in $I$.

The \textsc{CoverageUnion} procedure called from the \textsc{FindOne} algorithm simply returns the
union of the sets of requirements covered by test cases $i$ and $t$.

The \textsc{FindPossible} procedure called from the \textsc{FindOne} algorithm constructs candidate
combined test cases from two input test cases $i$ and $t$. We currently utilize two very simple heuristics for
test case combination as a part of this initial study:

\begin{enumerate}
	\item \textbf{Append}: Return a new test case $c$ formed by appending $t$ to $i$
	\item \textbf{AppendAfterInitial}: Determine the longest initial subsequence $s$ shared by $i$ and $t$.
	Return a new test case $c$ formed by appending $t - s$ to $i$
\end{enumerate}

The \textsc{GetProbFeasible} procedure called from \textsc{FindOne} returns an estimate of the
probability that a test case $c$ is feasible. We estimate probabilities with an N-gram model learned
from the execution of the initial input suite. Under this model, a sequence is considered to be 
\textit{more feasible} if the N-grams of the sequence have been observed in the input suite. Unobserved
sequences are still assigned some probability because we use a smoothing technique (in particular,
Kneser-Ney smoothing).

Similarly, the \textsc{GetProbCovered} procedure called from the \textsc{FindOne} algorithm returns an
estimate of the probability that all requirements in a set of requirements $V$ are covered by a test case
$c$. We compute this probability as the joint probability of each requirement $v$ from $V$ being met
by the test case. We estimate the probability that a test case $c$ covers a requirement $v$ from an N-gram
model specific to the requirement $v$, such that a probability returned by the model is the probability that the
given sequence covers the given requirement. As in the feasibility case, we learn the N-gram models from
observing per-event coverage of requirements during the execution of the initial suite.

\begin{figure}
\begin{algorithmic}[1]
\Procedure{Reduce}{$S$, $\alpha$, $\beta$}
\State $R = \{\}$
\State $I = S$
\While{$size(I) > 0$}
\State $i \gets $ \Call{SelectRandomSequence}{$I$}
\State $(c,s) \gets$ \Call{FindOne}{$i$, $I$, $\alpha$, $\beta$}
\If{$c \neq null$}
\State $R \gets R \cup c$
\State $I \gets I - s$
\Else
\State $R \gets R \cup i$
\EndIf
\EndWhile
\State \textbf{return} $R$
\EndProcedure
\end{algorithmic}
\caption{\textsc{Reduce} algorithm for producing a reduced test suite with feasibility and coverage confidence levels}
\label{a:reduce}
\end{figure}

\begin{figure}
\begin{algorithmic}[1]
\Procedure{FindOne}{$i$, $I$}
\ForAll{$t \in I$}
\State $V \gets $ \Call{CoverageUnion}{$i$, $t$}
\State $C \gets$ \Call{FindPossible}{$i$, $t$}
\ForAll{$c \in C$}
\State $P_f \gets$ \Call{GetProbFeasible}{$c$}
\State $P_c \gets$ \Call{GetProbCoverage}{$c$, $V$}
\If{$P_f \geq \alpha \land P_c \geq \beta$}
\State \textbf{return} $(c, t)$
\EndIf
\EndFor
\EndFor
\State \textbf{return} $(null, null)$
\EndProcedure
\end{algorithmic}
\caption{\textsc{FindOne} algorithm for finding a test case combination which meets the minimum confidence criteria}
\label{a:findone}
\end{figure}