% THIS IS SIGPROC-SP.TEX - VERSION 3.1
% WORKS WITH V3.2SP OF ACM_PROC_ARTICLE-SP.CLS
% APRIL 2009
%
% It is an example file showing how to use the 'acm_proc_article-sp.cls' V3.2SP
% LaTeX2e document class file for Conference Proceedings submissions.
% ----------------------------------------------------------------------------------------------------------------
% This .tex file (and associated .cls V3.2SP) *DOES NOT* produce:
%       1) The Permission Statement
%       2) The Conference (location) Info information
%       3) The Copyright Line with ACM data
%       4) Page numbering
% ---------------------------------------------------------------------------------------------------------------
% It is an example which *does* use the .bib file (from which the .bbl file
% is produced).
% REMEMBER HOWEVER: After having produced the .bbl file,
% and prior to final submission,
% you need to 'insert'  your .bbl file into your source .tex file so as to provide
% ONE 'self-contained' source file.
%
% Questions regarding SIGS should be sent to
% Adrienne Griscti ---> griscti@acm.org
%
% Questions/suggestions regarding the guidelines, .tex and .cls files, etc. to
% Gerald Murray ---> murray@hq.acm.org
%
% For tracking purposes - this is V3.1SP - APRIL 2009

\documentclass{acm_proc_article-sp}

\begin{document}

\title{Predicting the Feasibility of Unseen Test Cases}
%
% You need the command \numberofauthors to handle the 'placement
% and alignment' of the authors beneath the title.
%
% For aesthetic reasons, we recommend 'three authors at a time'
% i.e. three 'name/affiliation blocks' be placed beneath the title.
%
% NOTE: You are NOT restricted in how many 'rows' of
% "name/affiliations" may appear. We just ask that you restrict
% the number of 'columns' to three.
%
% Because of the available 'opening page real-estate'
% we ask you to refrain from putting more than six authors
% (two rows with three columns) beneath the article title.
% More than six makes the first-page appear very cluttered indeed.
%
% Use the \alignauthor commands to handle the names
% and affiliations for an 'aesthetic maximum' of six authors.
% Add names, affiliations, addresses for
% the seventh etc. author(s) as the argument for the
% \additionalauthors command.
% These 'additional authors' will be output/set for you
% without further effort on your part as the last section in
% the body of your article BEFORE References or any Appendices.

\numberofauthors{2} %  in this sample file, there are a *total*
% of EIGHT authors. SIX appear on the 'first-page' (for formatting
% reasons) and the remaining two appear in the \additionalauthors section.
%
\author{
% You can go ahead and credit any number of authors here,
% e.g. one 'row of three' or two rows (consisting of one row of three
% and a second row of one, two or three).
%
% The command \alignauthor (no curly braces needed) should
% precede each author name, affiliation/snail-mail address and
% e-mail address. Additionally, tag each line of
% affiliation/address with \affaddr, and tag the
% e-mail address with \email.
%
% 1st. author
\alignauthor
Bryan Robbins\\
       \affaddr{University of Maryland - College Park}\\
       \affaddr{Department of Computer Science}\\
       \affaddr{4115 AV Williams Building}\\
       \affaddr{College Park, MD 20742}\\
       \email{brobbins@cs.umd.edu}
% 2nd. author
\alignauthor
Atif Memon\\
       \affaddr{University of Maryland - College Park}\\
       \affaddr{Department of Computer Science}\\
       \affaddr{4115 AV Williams Building}\\
       \affaddr{College Park, MD 20742}\\
       \email{atif@cs.umd.edu}
}
\maketitle
\begin{abstract}
\input{parts/abstract}
\end{abstract}

\category{D.2.5}{Software Engineering}{Testing and Debugging}
\terms{Experimentation, Verification}
\keywords{Software testing, model-based testing, test automation}

\section{Introduction}
Model-based software testing excels at the generation of short test cases. Often,
the collective group of test cases (called the test suite) provides a guarantee
of coverage. Consider various existing work by Memon et al., which generates test
cases for event-driven systems such as those driven by Graphical User Interfaces
(GUIs). Their sequence length coverage generation technique experiences an
exponential growth of possible test cases as longer sequence lengths are considered. This
growth often leads to a practical limit of considering only sequences of length 2 or 3.
when generating test suites. Other research suggests that longer test cases, and test
cases involving the interaction of 3 or more variables, find signficantly more
bugs than test cases designed to cover only 1 or 2 input variables in combination.

We suggest that test cases generated from model-based techniques can potentially be
combined in interesting ways while still preserving the desired qualities of the overall
test suite. Significantly reducing the runtime of generated suites by combination would allow
for the consideration of increasingly higher-order test cases.

In support of this goal of combining test cases, in the paper we describe and
evaluate a method for predicting the feasibility of potential test case
combinations. We require our technique


\section{The {\secit Body} of The Paper}

\subsection{Citations}
Citations to articles \cite{bowman:reasoning, clark:pct, braams:babel, herlihy:methodology},
conference
proceedings \cite{clark:pct} or books \cite{salas:calculus, Lamport:LaTeX} listed
in the Bibliography section of your
article will occur throughout the text of your article.
You should use BibTeX to automatically produce this bibliography;
you simply need to insert one of several citation commands with
a key of the item cited in the proper location in
the \texttt{.tex} file \cite{Lamport:LaTeX}.
The key is a short reference you invent to uniquely
identify each work; in this sample document, the key is
the first author's surname and a
word from the title.  This identifying key is included
with each item in the \texttt{.bib} file for your article.

The details of the construction of the \texttt{.bib} file
are beyond the scope of this sample document, but more
information can be found in the \textit{Author's Guide},
and exhaustive details in the \textit{\LaTeX\ User's
Guide}\cite{Lamport:LaTeX}.

This article shows only the plainest form
of the citation command, using \texttt{{\char'134}cite}.
This is what is stipulated in the SIGS style specifications.
No other citation format is endorsed.

\subsection{Tables}
Because tables cannot be split across pages, the best
placement for them is typically the top of the page
nearest their initial cite.  To
ensure this proper ``floating'' placement of tables, use the
environment \textbf{table} to enclose the table's contents and
the table caption.  The contents of the table itself must go
in the \textbf{tabular} environment, to
be aligned properly in rows and columns, with the desired
horizontal and vertical rules.  Again, detailed instructions
on \textbf{tabular} material
is found in the \textit{\LaTeX\ User's Guide}.

Immediately following this sentence is the point at which
Table 1 is included in the input file; compare the
placement of the table here with the table in the printed
dvi output of this document.

\begin{table}
\centering
\caption{Frequency of Special Characters}
\begin{tabular}{|c|c|l|} \hline
Non-English or Math&Frequency&Comments\\ \hline
\O & 1 in 1,000& For Swedish names\\ \hline
$\pi$ & 1 in 5& Common in math\\ \hline
\$ & 4 in 5 & Used in business\\ \hline
$\Psi^2_1$ & 1 in 40,000& Unexplained usage\\
\hline\end{tabular}
\end{table}

To set a wider table, which takes up the whole width of
the page's live area, use the environment
\textbf{table*} to enclose the table's contents and
the table caption.  As with a single-column table, this wide
table will ``float" to a location deemed more desirable.
Immediately following this sentence is the point at which
Table 2 is included in the input file; again, it is
instructive to compare the placement of the
table here with the table in the printed dvi
output of this document.


\begin{table*}
\centering
\caption{Some Typical Commands}
\begin{tabular}{|c|c|l|} \hline
Command&A Number&Comments\\ \hline
\texttt{{\char'134}alignauthor} & 100& Author alignment\\ \hline
\texttt{{\char'134}numberofauthors}& 200& Author enumeration\\ \hline
\texttt{{\char'134}table}& 300 & For tables\\ \hline
\texttt{{\char'134}table*}& 400& For wider tables\\ \hline\end{tabular}
\end{table*}
% end the environment with {table*}, NOTE not {table}!

\subsection{Figures}
Like tables, figures cannot be split across pages; the
best placement for them
is typically the top or the bottom of the page nearest
their initial cite.  To ensure this proper ``floating'' placement
of figures, use the environment
\textbf{figure} to enclose the figure and its caption.

This sample document contains examples of \textbf{.eps}
and \textbf{.ps} files to be displayable with \LaTeX.  More
details on each of these is found in the \textit{Author's Guide}.

\begin{figure}
\centering
\epsfig{file=images/fly.eps}
\caption{A sample black and white graphic (.eps format).}
\end{figure}

\begin{figure}
\centering
\epsfig{file=images/fly.eps, height=1in, width=1in}
\caption{A sample black and white graphic (.eps format)
that has been resized with the \texttt{epsfig} command.}
\end{figure}


As was the case with tables, you may want a figure
that spans two columns.  To do this, and still to
ensure proper ``floating'' placement of tables, use the environment
\textbf{figure*} to enclose the figure and its caption.

Note that either {\textbf{.ps}} or {\textbf{.eps}} formats are
used; use
the \texttt{{\char'134}epsfig} or \texttt{{\char'134}psfig}
commands as appropriate for the different file types.

\subsection{Theorem-like Constructs}
Other common constructs that may occur in your article are
the forms for logical constructs like theorems, axioms,
corollaries and proofs.  There are
two forms, one produced by the
command \texttt{{\char'134}newtheorem} and the
other by the command \texttt{{\char'134}newdef}; perhaps
the clearest and easiest way to distinguish them is
to compare the two in the output of this sample document:

This uses the \textbf{theorem} environment, created by
the\linebreak\texttt{{\char'134}newtheorem} command:
\newtheorem{theorem}{Theorem}
\begin{theorem}
Let $f$ be continuous on $[a,b]$.  If $G$ is
an antiderivative for $f$ on $[a,b]$, then
\begin{displaymath}\int^b_af(t)dt = G(b) - G(a).\end{displaymath}
\end{theorem}

The other uses the \textbf{definition} environment, created
by the \texttt{{\char'134}newdef} command:
\newdef{definition}{Definition}
\begin{definition}
If $z$ is irrational, then by $e^z$ we mean the
unique number which has
logarithm $z$: \begin{displaymath}{\log e^z = z}\end{displaymath}
\end{definition}

\begin{figure}
\centering
\psfig{file=images/rosette.pdf, height=1in, width=1in,}
\caption{A sample black and white graphic (.ps format) that has
been resized with the \texttt{psfig} command.}
\end{figure}

Two lists of constructs that use one of these
forms is given in the
\textit{Author's  Guidelines}.

\begin{figure*}
\centering
\epsfig{file=images/flies.eps}
\caption{A sample black and white graphic (.eps format)
that needs to span two columns of text.}
\end{figure*}
and don't forget to end the environment with
{figure*}, not {figure}!
 
There is one other similar construct environment, which is
already set up
for you; i.e. you must \textit{not} use
a \texttt{{\char'134}newdef} command to
create it: the \textbf{proof} environment.  Here
is a example of its use:
\begin{proof}
Suppose on the contrary there exists a real number $L$ such that
\begin{displaymath}
\lim_{x\rightarrow\infty} \frac{f(x)}{g(x)} = L.
\end{displaymath}
Then
\begin{displaymath}
l=\lim_{x\rightarrow c} f(x)
= \lim_{x\rightarrow c}
\left[ g{x} \cdot \frac{f(x)}{g(x)} \right ]
= \lim_{x\rightarrow c} g(x) \cdot \lim_{x\rightarrow c}
\frac{f(x)}{g(x)} = 0\cdot L = 0,
\end{displaymath}
which contradicts our assumption that $l\neq 0$.
\end{proof}

Complete rules about using these environments and using the
two different creation commands are in the
\textit{Author's Guide}; please consult it for more
detailed instructions.  If you need to use another construct,
not listed therein, which you want to have the same
formatting as the Theorem
or the Definition\cite{salas:calculus} shown above,
use the \texttt{{\char'134}newtheorem} or the
\texttt{{\char'134}newdef} command,
respectively, to create it.

\subsection*{A {\secit Caveat} for the \TeX\ Expert}
Because you have just been given permission to
use the \texttt{{\char'134}newdef} command to create a
new form, you might think you can
use \TeX's \texttt{{\char'134}def} to create a
new command: \textit{Please refrain from doing this!}
Remember that your \LaTeX\ source code is primarily intended
to create camera-ready copy, but may be converted
to other forms -- e.g. HTML. If you inadvertently omit
some or all of the \texttt{{\char'134}def}s recompilation will
be, to say the least, problematic.

\section{Conclusions}
\input{parts/conclusion}

\bibliographystyle{abbrv}
\bibliography{amalga}
\balancecolumns
\end{document}
