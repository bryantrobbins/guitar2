
%% bare_jrnl_compsoc.tex
%% V1.3
%% 2007/01/11
%% by Michael Shell
%% See:
%% http://www.michaelshell.org/
%% for current contact information.
%%
%% This is a skeleton file demonstrating the use of IEEEtran.cls
%% (requires IEEEtran.cls version 1.7 or later) with an IEEE Computer
%% Society journal paper.
%%
%% Support sites:
%% http://www.michaelshell.org/tex/ieeetran/
%% http://www.ctan.org/tex-archive/macros/latex/contrib/IEEEtran/
%% and
%% http://www.ieee.org/

\documentclass[12pt,journal,compsoc,conf]{IEEEtran}

% *** CITATION PACKAGES ***
%
\ifCLASSOPTIONcompsoc
  % IEEE Computer Society needs nocompress option
  % requires cite.sty v4.0 or later (November 2003)
  % \usepackage[nocompress]{cite}
\else
  % normal IEEE
  % \usepackage{cite}
\fi

% *** GRAPHICS RELATED PACKAGES ***
%
\ifCLASSINFOpdf
	\usepackage[pdftex]{graphicx}
  % declare the path(s) where your graphic files are
  % \graphicspath{{../pdf/}{../jpeg/}}
  % and their extensions so you won't have to specify these with
  % every instance of \includegraphics
  % \DeclareGraphicsExtensions{.pdf,.jpeg,.png}
\else
\fi

% *** MATH PACKAGES ***
%
%\usepackage[cmex10]{amsmath}
% A popular package from the American Mathematical Society that provides
% many useful and powerful commands for dealing with mathematics. If using
% it, be sure to load this package with the cmex10 option to ensure that
% only type 1 fonts will utilized at all point sizes. Without this option,
% it is possible that some math symbols, particularly those within
% footnotes, will be rendered in bitmap form which will result in a
% document that can not be IEEE Xplore compliant!
%
% Also, note that the amsmath package sets \interdisplaylinepenalty to 10000
% thus preventing page breaks from occurring within multiline equations. Use:
%\interdisplaylinepenalty=2500
% after loading amsmath to restore such page breaks as IEEEtran.cls normally
% does. amsmath.sty is already installed on most LaTeX systems. The latest
% version and documentation can be obtained at:
% http://www.ctan.org/tex-archive/macros/latex/required/amslatex/math/





% *** SPECIALIZED LIST PACKAGES ***
%
\usepackage{algorithmic}

% *** ALIGNMENT PACKAGES ***
%\usepackage{array}
%\usepackage{mdwmath}
%\usepackage{mdwtab}
%\usepackage{eqparbox}

\ifCLASSOPTIONcompsoc
\usepackage[tight,normalsize,sf,SF]{subfigure}
\else
\usepackage[tight,footnotesize]{subfigure}
\fi

\ifCLASSOPTIONcompsoc
	\usepackage[caption=false]{caption}
	\usepackage[font=normalsize,labelfont=sf,textfont=sf]{subfig}
\else
	\usepackage[caption=false]{caption}
	\usepackage[font=footnotesize]{subfig}
\fi
\ifCLASSOPTIONcompsoc
	\usepackage[caption=false,font=normalsize,labelfont=sf,textfont=sf]{subfig}
\else
	\usepackage[caption=false,font=footnotesize]{subfig}
\fi
\usepackage{fixltx2e}
\usepackage{url}

% correct bad hyphenation here
%\hyphenation{op-tical net-works semi-conduc-tor}

\begin{document}
\title{Predicting the Feasibility of Potential Test Case Combinations}
\author{Bryan~Robbins and
        Atif~Memon,~\IEEEmembership{Member,~IEEE}\\
	Department of Computer Science\\
        University of Maryland\\
        College Park, MD 20742}

% The paper headers
%\markboth{Journal of \LaTeX\ Class Files,~Vol.~6, No.~1, January~2007}%

% for Computer Society papers, we must declare the abstract and index terms
% PRIOR to the title within the \IEEEcompsoctitleabstractindextext IEEEtran
% command as these need to go into the title area created by \maketitle.
\IEEEcompsoctitleabstractindextext{%
\begin{abstract}

\end{abstract}

% Note that keywords are not normally used for peerreview papers.
%\begin{IEEEkeywords}
%Computer Society, IEEEtran, journal, \LaTeX, paper, template.
%\end{IEEEkeywords}}


% make the title area
\maketitle

% To allow for easy dual compilation without having to reenter the
% abstract/keywords data, the \IEEEcompsoctitleabstractindextext text will
% not be used in maketitle, but will appear (i.e., to be "transported")
% here as \IEEEdisplaynotcompsoctitleabstractindextext when compsoc mode
% is not selected <OR> if conference mode is selected - because compsoc
% conference papers position the abstract like regular (non-compsoc)
% papers do!
\IEEEdisplaynotcompsoctitleabstractindextext
% \IEEEdisplaynotcompsoctitleabstractindextext has no effect when using
% compsoc under a non-conference mode.


% For peer review papers, you can put extra information on the cover
% page as needed:
% \ifCLASSOPTIONpeerreview
% \begin{center} \bfseries EDICS Category: 3-BBND \end{center}
% \fi
%
% For peerreview papers, this IEEEtran command inserts a page break and
% creates the second title. It will be ignored for other modes.
\IEEEpeerreviewmaketitle

\section{Introduction}
Model-based software testing excels at the generation of short test cases. Often,
the collective group of test cases (called the test suite) provides a guarantee
of coverage. Consider various existing work by Memon et al., which generates test
cases for event-driven systems such as those driven by Graphical User Interfaces
(GUIs). Their sequence length coverage generation technique experiences an
exponential growth of possible test cases as longer sequence lengths are considered. This
growth often leads to a practical limit of considering only sequences of length 2 or 3.
when generating test suites. Other research suggests that longer test cases, and test
cases involving the interaction of 3 or more variables, find signficantly more
bugs than test cases designed to cover only 1 or 2 input variables in combination.

We suggest that test cases generated from model-based techniques can potentially be
combined in interesting ways while still preserving the desired qualities of the overall
test suite. Significantly reducing the runtime of generated suites by combination would allow
for the consideration of increasingly higher-order test cases.

In support of this goal of combining test cases, in the paper we describe and
evaluate a method for predicting the feasibility of potential test case
combinations. We require our technique


\section{Model-Based Test Generation}
\input{parts/mbt}

\section{Modeling Feasibility}
\input{parts/feasible}

\section{Experiment}
Given the goals outlined above, we propose the following experiment as an
initial evaluation of our proposed feasibility prediction approach. We use
model-based GUI test suites as our input suites and Java Swing applications as
Applications Under Test (AUTs). To support execution of these suites, we
develop a scalable and portable execution infrastructure. Below, we elaborate
on the experimental setup and planned analyses.

/subsection{AUTs}

For this initial study, we chose two GUI-based Java Swing applications as
AUTs. While we acknowledge that choice of application platform and technology
may have some effects on experimental outcomes, in our case, Java tooling
for model-based test execution is far ahead of similar tooling for other
domains such as mobile and Web. We consider expansion of AUT domains an item
for future work.

Our first Java Swing application is the JabRef reference manager
\footnote{\url{http://jabref.sourceforge.net/}}, which features a complex
GUI built around extensive data entry and nearly 12 years of
open-source development. Our second application is
ArgoUML\footnote{\url{http://argouml.tigris.org/}}, a largely graphical
GUI-based Java application for constructing diagrams with connected
components. These applications represent real-world examples for model-based
testing. They also draw from multiple domains and use cases as GUI-based
applications.





\section{Results and Analysis}
\input{parts/results}

\section{Threats to Validity}
\input{parts/threats}

\section{Related Work}
\input{parts/related}

\section{Conclusions and Future Work}
\input{parts/conclusion}

\newpage

\begin{IEEEbiographynophoto}{Jane Doe}
Biography text here.
\end{IEEEbiographynophoto}

% You can push biographies down or up by placing
% a \vfill before or after them. The appropriate
% use of \vfill depends on what kind of text is
% on the last page and whether or not the columns
% are being equalized.

%\vfill

% Can be used to pull up biographies so that the bottom of the last one
% is flush with the other column.
%\enlargethispage{-5in}



% that's all folks
\end{document}
