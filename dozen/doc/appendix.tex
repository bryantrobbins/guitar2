

\appendix

\section{Appendix}
\label{sec:appendix}

Utilities and software required for using \mddozen{} with a testbed.

\subsection{Xvfb}

Xvfb\footnote{\url{http://www.xfree86.org/4.0.1/Xvfb.1.html}} is an X server that can run on machines with no display hardware. It emulates a dumb framebuffer using virtual memory.

\begin{figure}

   \scriptsize{
   \noindent Start Xvfb:\\
   \begin{lstlisting}
   Xvfb :<display_number> -fbdir <session_dir>
                          -screen 0 1600x1200x16
   \end{lstlisting}
   }

   \noindent Stop Xvfb:\\

   \scriptsize{
   \begin{lstlisting}
   pkill -x "(Xvfb)"
   rm -f /tmp/.X11-unix/X:<display_number>
   \end{lstlisting}
   }
   \caption{\texttt{Xvfb} commands.}
   \label{fig:xvfb}
\end{figure}

Commands to start and stop \texttt{Xvfb} are shown in Figure~\ref{fig:xvfb}. It uses the \texttt{$<$session\_dir$>$} as a scratch directory, uses the virtual display entry \texttt{$<$display\_number$>$} and creates a virtual screen of resolution $1600*1200$ with $16$-bit colour depth.

When started Xvfb creates two lock files --
\begin{lstlisting}
/tmp/.X<display_number>-lock
/tmp/.x11-unix/X<display_number>
\end{lstlisting}
Xvfb will not start if these files are already present. After stopping \texttt{Xvfb} these lock file, if present, needs to be deleted.


\subsection{Jenkins}

Jenkins\footnote{http://jenkins-ci.org/} is a build and job control system. Jenkins allows jobs to be executed locally on a machine as well on a remote machines which are configured as a Jenkins slave.

When used with \mddozen{} it is recommended that simple Jenkins jobs be created to run on a local machine only. The Jenkins job itself can execute scripts on remote machines using \texttt{ssh}. This gives greater control to the job scripts. For example, Figure~\ref{fig:jenkinsjob} shows the command line from a simple Jenkins job.


\begin{figure}
   \scriptsize{
   \begin{lstlisting}
   ./dozen/jenkins/cluster-run-testsuite.sh
      ArgoUML
      jenkins gandalf.cs.umd.edu /tmp/auts/
      sq_l_2
      ./dozen/jenkins/experiments-one-machine.cfg
   \end{lstlisting}
   }

   \caption{A simple Jenkins job. The script \texttt{cluster-run-testsuite.sh} has full control on the cluster.}
   \label{fig:jenkinsjob}
\end{figure}


\subsection{.bashrc}
\label{sec:bashrc}

The controller and archive machine requires \texttt{dozen/} and \texttt{ecig/} scripts from time to time. It would be useful to checkout and store a copy of the \texttt{dozen/} and \texttt{ecig/} repositories on these two machines. It would be also useful to have certain environment variables set up in both places.

Figure~\ref{fig:bashrc} shows the example of a \texttt{$\tilde{}$/.bashrc} file containing environment settings. The path to Java binaries or JDK may also need to be appended to \texttt{PATH}.


\begin{figure}
   \scriptsize{
   \begin{lstlisting}
# History
export HISTSIZE=1048576
export HISTCONTROL=erasedups
shopt -s histappend

# ANT
export ANT_HOME=/usr/bin/ant/
export PATH=$PATH:/usr/bin/ant/bin/

# CSSVN
export CSSVN=/root/
export SVN_EDITOR=vi

# DOZEN
export DOZEN=$CSSVN/dozen
export PATH=$PATH:$DOZEN/tools/scripts:
            $DOZEN/tools/bin/guitar/bin:
            $DOZEN/tools/bin/cobertura:
            $DOZEN/common/

# ECIG
export ECIG=$CSSVN/ecig
export PATH=$PATH:$ECIG/bin/
export PERL5LIB=$ECIG/bin/

# Java
export JAVA_HOME=/usr/java/jdk1.7.0/
export PATH=$PATH:$JAVA_HOME

# Misc
alias dozen='cd $CSSVN/dozen'
alias ecig='cd $CSSVN/ecig'
   \end{lstlisting}
   }

   \caption{A \texttt{$\tilde{}$/.bashrc} file to be used on controller and archive machine.}
   \label{fig:bashrc}
\end{figure}

\subsection{ssh-keygen}
\label{sec:sshkeygen}

To enable password-less \texttt{ssh}, \texttt{rsync} from machine local-host to machine remote-host, the steps described in Figure~\ref{fig:sshkeygen} are done.

\begin{figure}
   \scriptsize{
      \begin{lstlisting}
local-host:

   ssh-keygen -t rsa      # no passphrase
   scp ~/.ssh/id_rsa.pub <remote-user>@<remote-host>:/tmp/

remote-host:

   cat /tmp/id_rsa.pub >> ~/.ssh/authorized_keys
   \end{lstlisting}
   }
   \caption{Password-less \texttt{ssh} from local-host to remote-host.}
   \label{fig:sshkeygen}
\end{figure}
