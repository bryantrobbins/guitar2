

\section{Introduction}
\label{sec:introduction}

Maryland dozen is a set of twelve Java applications, a test harness, a set of tools and user-defined workflows. These form a system for evaluating and demonstrating GUITAR, its features and algorithms. This document describes Maryland dozen and provides instructions for using it.

The top-level directory structure of the \mddozen{} is shown in Figure~\ref{fig:dozenlayout}. In this figure, the directory \texttt{dozen} is the root directory for the \mddozen{}. It can be checked out from the repository at:

\begin{center}
\begin{equation}
http://svn.cs.umd.edu/repos/guitar/dozen
\end{equation}
\end{center}

The \texttt{auts/} directory contains applications, application specific scripts and configurations. The \texttt{common/} directory contains all scripts forming the test harness. They are used for performing different testing steps on the applications. The \texttt{tools} directory contain tools such as GUITAR and scripts required to use them.

Section~\ref{sec:components} describes different components of the Maryland dozen. Section~\ref{sec:workflow} describes workflows for conducting experiments. Section~\ref{sec:process} shows certain possible ways of analysing the artifacts produced by \mddozen{}.

\begin{figure}
\scriptsize{
   \begin{lstlisting}[language=sh, showstringspaces=false, breaklines=true]
      dozen/
        +- auts/
        +- common/
        +- tools/
        +- jenkins/
   \end{lstlisting}
}
   \caption{Top-level directory structure of \mddozen{}. All directories are obtained from the repository.}
   \label{fig:dozenlayout}
\end{figure}


